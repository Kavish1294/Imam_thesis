\chapter*{Abstract}\label{ch:abstract}
%The nEDM experiment at the Spallation Neutron Source (nEDM@SNS) will implement a novel scheme which utilizes the unique properties of combining polarized ultracold neutrons (UCN), polarized $^3$He, and superfluid $^4$He to place a new limit on the nEDM down to 10$^{-28}$ e$\cdot$cm. The experiment will employ a cryogenic magnet to provide the required magnetic field environment to achieve the proposed sensitivity. The polarized cold neutron beam will pass through the cryogenic magnet to reach the superfluid $^4$He inside the measurement cells in order to produce polarized UCNs. This dissertation describes the design and implementation of a $^3$He polarimetry setup at the SNS to measure the neutron polarization loss and transmission efficiency through the cryogenic magnet windows. Results from monochromatic neutron polarization measurements as well as polarimetry component optimization will be presented.

%The neutron electric dipole moment experiment at the Spallation Neutron Source (nEDM@SNS) will implement a novel method, which utilizes polarized ultra-cold neutrons (UCN) and polarized $^3$He in a bath of superfluid $^4$He, to place a new limit on the nEDM down to 2-3$\times$10$^{-28}$ e·cm. The experiment will employ a cryogenic magnet and magnetic shielding package to provide the required magnetic field environment to achieve the proposed sensitivity. This dissertation describes the design and implementation of a polarized $^3$He based neutron polarimetry setup at the SNS to measure the monochromatic neutron polarization and transmission losses resulting from passage through the magnetic shielding and cryogenic windows. Results from monochromatic neutron polarization and transmission measurements as well as optimization of neutron polarimetry components are presented.

The D.O.E Nuclear Science Advisory Committee Long Range Plan has called for experimental programs to explore fundamental symmetry violations and their implications in nuclear, particle and cosmological physics. The neutron electric dipole moment experiment at the Spallation Neutron Source (nEDM@SNS) aims to search for new physics in the Time-reversal (T) and Charge-Parity (CP) symmetry violating sector by setting a new limit on the nEDM down to a few $\times~10^{-28}$ e$\cdot$cm using a novel cryogenic technique, which combines the unique properties of polarized Ultracold Neutrons (UCN), polarized $^3$He, and superfluid $^4$He. The experiment will employ a cryogenic magnet and magnetic shielding package to provide the magnetic field environment required to achieve the proposed sensitivity. This dissertation describes the design and implementation of a $^3$He based neutron polarimetry setup at the SNS to measure the monochromatic neutron polarization and transmission losses resulting from passage through the magnetic shielding and cryogenic windows. Results from monochromatic neutron polarization and transmission measurements will be presented. The work described will verify the design of the cryogenic magnet and allow the nEDM@SNS experiment to initiate assembly and commissioning for physics data collection.