\chapter{Discussions and Conclusions}
\label{ch:conclusions}

\ifpdf
    \graphicspath{{Chapter5/Figs/Raster/}{Chapter5/Figs/PDF/}{Chapter5/Figs/}}
\else
    \graphicspath{{Chapter5/Figs/Vector/}{Chapter5/Figs/}}
\fi

The nEDM@SNS experiment aims to place a new upper limit on the nEDM down to a few $\times 10^{-28}$ e$\cdot$cm. This new upper limit has the power to constrain BSM CP violating theories that propose explanations for the baryon asymmetry of the observed universe. The nEDM@SNS experiment plans to reach it's proposed sensitivity goals by using an innovative technique based UCNs and $^3$He in superfluid $^4$He. This technique requires a meticulous control over the magnetic fields, and the nEDM@SNS experiment's cryomagnet will be used to achieve this. 

%Simulation of neutron polarization transitions through magnetic materials are difficult therefore, polarization loss has to be measured experimentally. $^3$He polarimetry is commonly utilized for neutron polarization analysis, therefore, becomes an ideal choice for the polarization and transmission measurements. 

This thesis describes a polarized $^3$He based neutron polarimetry experimental setup to measure the neutron polarization and transmission loss from the nEDM@SNS experiment's cryogenic magnet. A proof of principle of was demonstrated with a neutron polarimetry measurement on a new polarized monochromatic neutron beamline using a compact in situ $^3$He spin analyser system. The polarization and transmission measurements conducted measured a neutron polarization upstream of the nEDM cryomagnet, however, a full neutron depolarization, as the polarized neutron beam traversed through the nEDM cryomagnet, was observed. 

Since the beam polarization is defined by the guiding magnetic fields, the obvious cause of this depolarization are the spin transport magnetic fields as well as the magnetic fields created by the cryomagnet system. \Cref{fig:measure_mag} of \cref{app:down_poldata} shows the measured magnetic field profiles along the beamline in the upstream and downstream locations of the cryomagnet. They indicate low guide magnetic field zones near the mu-metal shield of the cryomagnet apparatus. The dominant field in these regions should be in the horizontal direction. The lack of such a guiding magnetic field from guiding magnetic field components was caused by the physical limitations of placing guiding magnetic field components as well as the magnetic flux capture by the high permeability mu-metal MSE shielding surrounding the cryomagnet. This allowed for the stray magnetic fields to depolarize the neutrons, adiabatically, near the mu-metal beam entry port. The magnetic fields from the cryomagnet $B_0$ coil were also misaligned as the $B_0$ coil was clocked, creating a magnetic field in the beam axis in addition to the horizontal axis. For the future, the plan is to align the cryomagnet $B_0$ coil and mu-metal shield to their ideal positions and repeat the polarization and transmission measurements. The magnetic field profile will need to be verified with 3-axis flux gate magnetometery to ensure the presence of guiding magnetic fields all along the polarized neutron flight path especially near the mu-metal shield. The probe array will also be used verify the $B_0$ magnetic field inside the cryomagnet. 

The $^3$He polarization of the cell Soccer as measured via neutron transmission was low as compared to the expected $^3$He polarization from previous EPR measurements on the $^3$He cell Soccer. This is attributed to a possible misalignment of the laser optical components that are responsible for producing the circularly polarised light during the transfer of the in situ system from the laser laboratory to the beamline experimental hall. As mentioned in \cref{ch:polHe}, misalignment of laser optics will produce an offset in the D1 resonance optical pumping, limiting the efficiency of the SEOP process and hence, limiting the maximally attainable $^3$He polarization. Due to the limited size of the in situ system's heating oven, the cell Soccer fits in the heating oven only in a certain orientation. It is also possible that this is the non-optimal SEOP orientation of $^3$He based on the $^3$He cell orientation effect described in \cref{ch:polHe}. 

There may be systematic effects arising from the transfer of polarized $^3$He cell to upstream to downstream of the cryomagnet to measure the neutron polarization loss. In hindsight, the neutron polarization loss from the cryomagnet should have been measured with the polarized $^3$He spin analyzer always kept in the downstream location and the the cryomagnet introduced into the neutron flight path as a retractable ``sample". Due to the infrastructure and logistical challenges associated with treating the cryomagent as a retractable sample, this approach was abandoned. In the experimental scheme for measuring polarization as shown in \cref{fig:polarimetry_setup}, it was assumed that neutron polarization was uniform across the cross sectional area of the beam. It would be worth testing this assumption by measuring the neutron polarization across the $^3$He analyzer cell. The neutron beam profile is much wider than the diameter of the $^3$He cell, therefore only the relevant part of the beam traverses through the cell. A four jaw collimator with two horizontal and two vertical neutron absorbing plates that could be adjusted independently can be used to analyze the neutron polarization across the $^3$He cell as a grid \cite{McCrea2020}.

Since the underlying quantities being measured to determine the neutron beam polarization are the transmission measurements through various configurations, any changes in neutron transmission from sources other than the intended configuration setup can lead to a systematic shift in the actual transmission and subsequently, cause a systematic uncertainty in the neutron beam polarization extraction. A low efficiency neutron detector upstream of the polarizer can be used to account for beam fluctuations due to neutron source instability. However, because the precision level, for which the neutron polarization and transmission loss need to measured, is very low, corrections for the higher order effects may not be needed.

The monochromatic beamline 13A also produces $\lambda$/n wavelengths. It will be a useful check to measure the neutron polarization using the $^3$He cell, Soccer, at 4.5~\AA\ wavelength. Since the neutron $^3$He capture reaction is wavelength dependent, measuring neutron polarization at 4.5~\AA\ will diagnose the analyzing power of Soccer, assuming all other cell parameters are held constant. Similarly, the neutron intensity loss from the neutron absorption in the beam window materials is also wavelength dependent. The transmission loss measurement at 4.5~\AA\ will verify the transmission loss at 8.9~\AA\ measured in this thesis, since the transmission will scale based on the neutron absorption cross at the respective wavelengths.  

The SMP neutron polarizer was crudely positioned with respect to the flight tube. A finer positioning of the SMP utilizing feedback from position indicators may help further maximize the polarized neutron yield from the SMP output. A possible redesign of the experimental setup can be done by using the polarized $^3$He as the neutron polarizer and the SMP as the neutron spin analyzer. This avoids the problem of a diverging neutron beam emerging from the SMP, since $^3$He does not diverge the transmitted neutron beam. Furthermore, the analyzing power of the SMP is much higher, therefore, a larger neutron spin contrast can be observed, albeit limited by the polarizing power of the $^3$He cell. The analysis will also be made more robust by developing an event based data acquisition, which includes the pulse height ADC value for each neutron event in addition to the time of flight and positional information.

The transmission of 8.9~\AA\ neutrons through the various beam window materials to characterize the neutron beam intensity loss was also measured. A lower transmission through the cryomagnet was measured as compared to the expected. This is attributed to the fact the neutron beam had a large horizontal divergence and scattering off of the beam window frames. Further studies need to be performed with a tightly collimated neutron beam, for e.g. by using the four jaw collimators as described earlier, to eliminate the beam window frame scattering and obtain a more accurate neutron intensity loss from the beam windows. 

The SANS measurements as well as beam intensity loss measurements were performed on the candidate beam window materials for the nEDM@SNS experiment. The SANS measurement on all samples showed that the beam divergence due to SANS from the window materials was smaller than the angular divergence acceptance of the measurement cell. This means that these materials will not cause the neutron beam to interact with off beam axis components of the nEDM@SNS experiment and compromise the expected signal to background ratio of the experiment. These measurements also showed that the beam intensity loss from these window materials was small and agreed with the expectation. In spite of this, a variation on the neutron beam intensity loss on the Metglas sample was observed due to neutron beam alignment issues. The expected Metglas beam intensity loss was also difficult to calculate due to uncertainty in the percent mass composition of the Boron in Metglas. Neutron activation analysis will need to be performed to accurately determine the impurity levels and the mass composition of the window material. This will be used to characterize the activity of each of the window materials to ensure that the expected signal-to-background ratio of the experiment is not compromised.

Performing the upgrades outlined above will allow for a more accurate determination of the neutron polarization after the cryomagnet and thus, the contribution of the cryomagnet toward neutron polarization loss can be determined. These results will verify the cryomagnet design and advance the nEDM@SNS experiment towards assembly and commissioning for physics data taking to place a constraint on the nEDM as an additional source of T/CP symmetry violation necessary to explain the baryon asymmetry of the universe.