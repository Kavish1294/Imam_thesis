\chapter{Propagation of Uncertainty} \label{app:error}

The polarization is defined in \cref{eq:polarization} as:

\begin{equation}
    P_n = \frac{R- R_{sf}}{\sqrt{\left( \left(2\epsilon_{sf}-1 \right)R+R_{sf}\right)^2-4\epsilon_{sf}^2}}
\end{equation}
The propagation of uncertainty for the neutron polarization can be done by:
\begin{equation}
    \sigma_P^2 = \sigma_R^2 \left(\frac{\partial P}{\partial R}\right)^{2}+ \sigma_{R_{sf}}^2 \left(\frac{\partial P}{\partial R_{sf}}\right)^{2} + \sigma_{\epsilon_{sf}}^2 \left(\frac{\partial P}{\partial \epsilon_{sf}}\right)^{2}
\end{equation}

The cross-correlation terms are being neglected due the fact that the measurements are independent of each other i.e. each transmission measurement was taken separately.
\begin{align}
    \frac{\partial P}{\partial R} 
    &= \frac{2 \epsilon_{sf} \left(R_{sf} \left(-R + R_{sf}\right) + 2 \epsilon_{sf} \left(-1 + R R_{sf}\right)\right)}{\left(-4 \epsilon_{sf}^2 + \left(\left(-1 + 2 \epsilon_{sf}\right) R + R_{sf}\right)^2\right)^{3/2}} \nonumber \\
    &=\frac{4RR_{sf}\epsilon_{sf}^2-2RR_{sf}\epsilon_{sf}+2R_{sf}^2\epsilon_{sf}-4\epsilon_{sf}^2}{\left( \left(\left(2 \epsilon_{sf} - 1 \right) R + R_{sf}\right)^2 - 4 \epsilon_{sf}^2 \right)^{3/2}}
\end{align}
\begin{align}
    \frac{\partial P}{\partial R_{sf}} 
    & = \frac{2 \epsilon_{sf} \left(-2 \epsilon_{sf} \left(-1 + R^2\right) + R \left(R - R_{sf}\right)\right)}{\left(-4 \epsilon_{sf}^2 + \left(\left(-1 + 2 \epsilon_{sf}\right) R + R_{sf}\right)^2\right)^{3/2}} \nonumber \\
    & = \frac{-4R^2\epsilon_{sf}^2+2R^2\epsilon_{sf}-2RR_{sf}\epsilon_{sf}+4\epsilon_{sf}^2}{\left(\left(\left(2 \epsilon_{sf} - 1 \right) R + R_{sf}\right)^2 - 4 \epsilon_{sf}^2\right)^{3/2}}
\end{align}
\begin{align}
    \frac{\partial P}{\partial \epsilon_{sf}} 
    & = \frac{-2 \left(R - R_{sf}\right) \left(-2 \epsilon_{sf} - R^2 + 2 \epsilon_{sf} R^2 + R R_{sf}\right)}{\left(-4 \epsilon_{sf}^2 + \left(\left(-1 + 2 \epsilon_{sf}\right) R + R_{sf}\right)^2\right)^{3/2}} \nonumber \\
    & = \frac{-4R^3\epsilon_{sf}-2R^3+4R^2R_{sf}\epsilon_{sf}-4R^2R_{sf}+2RR_{sf}^2+4R\epsilon_{sf}-4R_{sf}\epsilon_{sf}}{\left(\left(\left(2 \epsilon_{sf} - 1 \right) R + R_{sf}\right)^2 - 4 \epsilon_{sf}^2\right)^{3/2}}
\end{align}
and
\begin{equation}
    \sigma_R^2 = R^2\left(\left(\frac{\sigma_T}{T}\right)^2+\left(\frac{\sigma_{T_0}}{T_0}\right)^2\right)
\end{equation}
\begin{equation}
    \sigma_{R_{sf}}^2 = R_{sf}^2\left(\left(\frac{\sigma_{T_{sf}}}{T_{sf}}\right)^2+\left(\frac{\sigma_{T_0}}{T_0}\right)^2\right)
\end{equation}
\begin{equation}
    \sigma_{\epsilon_{sf}}^2 = \sigma_T^2 \left(\frac{\partial \epsilon_{sf}}{\partial T}\right)^{2}+ \sigma_{T_{sf}}^2 \left(\frac{\partial \epsilon_{sf}}{\partial T_{sf}}\right)^{2} + \sigma_{T^{AFP}}^2 \left(\frac{\partial \epsilon_{sf}}{\partial T^{AFP}}\right)^{2} + \sigma_{T_{sf}^{AFP}}^2 \left(\frac{\partial \epsilon_{sf}}{\partial T_{sf}^{AFP}}\right)^{2}
\end{equation}
where,
\begin{equation}
    \frac{\partial \epsilon_{sf}}{\partial T} = \frac{-(T^{AFP} (T_{sf} + T_{sf}^{AFP}))}{((T + T^{AFP})^2 (T_{sf} - T_{sf}^{AFP}))}
\end{equation}
\begin{equation}
    \frac{\partial \epsilon_{sf}}{\partial T_{sf}} = \frac{(T (T_{sf} + T_{sf}^{AFP}))}{((T + T^{AFP})^2 (T_{sf} - T_{sf}^{AFP}))}
\end{equation}
\begin{equation}
    \frac{\partial \epsilon_{sf}}{\partial T^{AFP}} = \frac{((T - T^{AFP}) T_{sf}^{AFP})}{((T + T^{AFP}) (-T_{sf} + T_{sf}^{AFP})^2)}
\end{equation}
\begin{equation}
    \frac{\partial \epsilon_{sf}}{\partial T_{sf}^{AFP}} = \frac{(T_{sf} (-T + T^{AFP}))}{((T + T^{AFP}) (-T_{sf} + T_{sf}^{AFP})^2)}
\end{equation}